\documentclass{article}
\usepackage[a4paper]{geometry}
\usepackage{hyperref}
\usepackage{syntax}
\usepackage{listings}
\lstset{language=C}
\usepackage{parskip}
\input{ebnf.sty}

\hypersetup{
    colorlinks,
    citecolor=black,
    filecolor=black,
    linkcolor=black,
    urlcolor=black
}

\author{Christian Manning -- p0928544x\\
De Montfort University}
\title{IMAT3451 -- Final Year Project\\
    Dynamic Data Structure Visualisation\\
    User Documentation
}
\date{February 2012}

\begin{document}
\maketitle

\tableofcontents

\pagebreak

\section{Introduction}

The Dynamic Data Structure Visualisation (DDSV) application aims to provide help to students who are learning to program. The focus of this application is on the concept of pointers; specifically, how they are used to build data structures.

It does this by utilising a graphical user interface (GUI) to provide a visualisation containing graphical representations of abstract data types and pointers to them. This facilitates the visual construction of a data structure, which can then be manipulated to simulate algorithmic operations.

Another interface to this application utilises the expressiveness of a programming language from the GUI in the form of an interpreter. The language accepted by this interpreter is a subset of the C programming language, defined later in this document, which uses quite a low level model and so pointers are exposed to the user more readily.

\section{User Interface}

\section{Interpreted Language}

\subsection{Specification}

This is the formal grammar specification of the interpreted language using an Extended Backus-Naur Form (EBNF)-like notation, with some brief descriptions.

\noindent\textit{binary\_op} ::=
    '$||$'
    $|$ '$\&\&$'
    $|$ '$==$'
    $|$ '$!=$'
    $|$ '$<$'
    $|$ '$\leq$'
    $|$ '$>$'
    $|$ '$\geq$'
    $|$ '$+$'
    $|$ '$-$'
    $|$ '$*$'
    $|$ '$/$'
    ;

\noindent These are \textit{binary operators}, meaning they require two \textit{operands}.

\noindent\textit{unary\_op} ::=
    '$+$'
    $|$ '$-$'
    $|$ '$!$'
    $|$ '$*$'
    $|$ '$\rightarrow$'
    $|$ '$.$'
    $|$ '$\&$'
    ;

\noindent\textit{Unary operators} require only one \textit{operand}.

\noindent\textit{keywords} ::=
    'true' $|$ 'false' $|$ 'if' $|$ 'else' $|$ 'while' $|$ 'struct' $|$ 'return'
    ;

\noindent These are the \textit{keywords} which are reserved words and cannot be used outside of their required context.

\noindent\textit{types} ::=
    'void' $|$ 'int' $|$ 'bool' ;
    
\noindent These are the predefined primitive \textit{types} that are used for declaring variables, etc.

\begin{grammar}
<expr> ::=
    <unary\_expr> {<binary\_op> <unary\_expr>}
    ;
\end{grammar}
An \textit{expression} must contain at least one \textit{unary expression}, with an option for any number of \textit{binary} operations paired with another \textit{unary expression} appended.
\begin{grammar}
<unary\_expr> ::=
	<primary\_expr>
    \alt (<unary\_op> <primary\_expr>)
    ;
\end{grammar}
A \textit{unary expression} consists of either a \textit{primary expression} or a \textit{unary operator} followed by a \textit{primary expression}.
\begin{grammar}
<identifier> ::=
	-(keywords | types) ,
	alpha | '\_' , \{ alpha | digit | '\_' \} ;
\end{grammar}
An \textit{identifier} must not be a \textit{keyword}, a primitive type or begin with a digit. It must start with an alphabetic character or an underscore (\_) to be optionally followed by zero or more alphabetic characters, digits or underscores. 
\begin{grammar}
<primary\_expr> ::=
	integer
    \alt   <function\_call>
    \alt   <identifier>
    \alt   boolean
    \alt   '(' <expr> ')'
;
\end{grammar}
Apart from the other specified \textit{expressions}, a \textit{primary expression} can also be an integer literal, a boolean literal (\textit{true} \& \textit{false}), or an expression enclosed in parentheses ('(' \& ')').
\begin{grammar}
<function\_call> ::=
<identifier> '(' [<call_argument_list>] ')'
;
\end{grammar}
\begin{grammar}
<call_argument\_list> ::= [<expr> | <expr> ',' <call_argument_list>] ;
\end{grammar}
A \textit{call argument list} can contain zero, one or many expressions.
\begin{grammar}
<statement_list> ::=
    \{statement_\}
    ;
\end{grammar}
\begin{grammar}
<statement_> ::=
	<variable_declaration>
    \alt   <struct_member_declaration>
    \alt   <struct_declaration>
    \alt   <struct_instantiation>
    \alt   <assignment>
    \alt   <compound_statement>
    \alt   <function_call_statement>
    \alt   <if_statement>
    \alt   <while_statement>
    \alt   <return_statement>
    ;
\end{grammar}
Any of the above listed \textit{statements} can take the place of a $\langle statement\_\rangle$ instance.
\begin{grammar}
<type_id> ::= <identifier> ;
\end{grammar}
\begin{grammar}
<var_type> ::= <types> ['*'] ;
\end{grammar}
The optional '*' denotes a pointer type.
\begin{grammar}
<variable_declaration> ::=
	<var_type> <identifier> [('=' <expr>)] ';' ;
\end{grammar}
\begin{grammar}
<struct_member_declaration> ::=
	(<var_type> | <struct_instantiation>) <identifier> ';' ;
\end{grammar}
\begin{grammar}
<struct_declaration> ::=
	'struct' <type_id>
    '\{'
    \{<struct_member_declaration>\}
    '\}'
    ';'
    ;
\end{grammar}
\begin{grammar}
<struct_instantiation> ::=
	'struct'
    <type_id>
    <identifier>
    ';'
    ;
\end{grammar}
\begin{grammar}
<assignment> ::=
	<identifier>
       '='
       <expr>
       ';'
    ;
\end{grammar}
\begin{grammar}
<if_statement> ::=
	'if'
       '('
       <expr>
       ')'
       <statement_>
	[
	    'else'
	   <statement_>
	]
    ;
\end{grammar}
\begin{grammar}
<while_statement> ::=
	'while'
       '('
       <expr>
       ')'
       <statement_>
    ;
\end{grammar}
\begin{grammar}
<compound_statement> ::=
    '\{' [<statement_list>] '\}'
    ;
\end{grammar}
A \textit{compound statement} is an optional list of \textit{statements} enclosed in braces ('\{' \& '\}').
\begin{grammar}
<return_statement> ::=
	'return'
    [expr]
       ';'
    ;
\end{grammar}
\begin{grammar}
<argument_list> ::= \{<var_type> <identifier>\} ;
\end{grammar}
\begin{grammar}
<function_declaration> ::=
	<var_type>
       <identifier>
       '(' <argument_list> ')'
       <compound_statement>
    ;
\end{grammar}
\begin{grammar}
<translation_unit> ::=
    \{ <statement_list> | <function> \} ;
\end{grammar}
The \textit{translation unit} consists of a list of both \textit{function} declarations and \textit{statement lists}. This is root of all that can be passed to the interpreter, i.e.\ everything must be entered in this form.

\subsection{Operator Precedence}
The following table shows the operator precedence of the interpreted language, from lowest to highest.
\begin{tabular}{| c | l | l |}
\hline
Precedence & Operator & Description \\ \hline
1 & , & Comma \\ \hline
2 & = & Assignment \\ \hline
3 & $\parallel$ & Logical OR \\ \hline
4 & \&\& & Logical AND \\ \hline
5 & == & Equal \\
  & != & Not Equal \\ \hline
6 & $<$ & Less than \\
  & $\leq$ & Less than or equal to \\
  & $>$ & More than \\
  & $\geq$ & More than or equal to \\ \hline
7 & + (binary) & Addition \\
  & $-$ (binary) & Subtraction \\ \hline
8 & * & Multiplication \\
  & / & Division \\ \hline
9 & + (unary) & Plus \\
  & -- (unary) & Minus \\
  & ! & Not \\
  & $\&$ & Address of \\
  & * (unary) & Dereference \\ \hline
10 & $-$$>$ & Select element through pointer \\
   & . & Select element \\ \hline

\end{tabular}

\subsection{Examples}

\subsubsection{Variable Declaration}

\begin{lstlisting}
int a; //declare an integer variable named "a"
int b = 3; //assignment on declaration
int c = b + 2; //assignment & binary operation

int * p = &a;
*p = 51; // a == 51
\end{lstlisting}

\subsubsection{Function Definition}

\begin{lstlisting}
int factorial(int n) {
  if(n < 1)
    return 1;
  else
    return n * factorial(n-1);
}
\end{lstlisting}


\end{document}
